% my template for CS70 Spring 2019
\documentclass{article}
\usepackage{enumitem}
\usepackage{amsmath}
\usepackage{amssymb}
\usepackage{textcomp}
\usepackage{graphicx}
\usepackage[mathscr]{euscript}
\let\euscr\mathscr \let\mathscr\relax% just so we can load this and rsfs
\usepackage[scr]{rsfso}
\newcommand{\powerset}{\raisebox{.15\baselineskip}{\Large\ensuremath{\wp}}}
% for making real R's
\newcommand{\R}{\mathbb{R}}
% for making really rational Q's
\newcommand{\Q}{\mathbb{Q}}
% for making integerZ
\newcommand{\Z}{\mathbb{Z}}
% for making really natural N's
\newcommand{\N}{\mathbb{N}}
% for making cool P's, probably
\newcommand{\PP}{\mathbb{P}}
% for making E's, I expect?
\newcommand{\E}{\mathbb{E}}
% for keeping tabs
\newcommand\tab[1][1cm]{\hspace*{#1}}

% for titles, i.e. CS70 Spring 3043 HW1024
\title{CS 70 Semester YYYY HW 00}

% for authors
\author{Author Name}

% we begin, with a document
\begin{document}

\maketitle

\section*{Sundry}

\section{A Title}

% enumeration via lower case letters
\begin{enumerate}[label=(\alph*)]
\item{
    Some CS70 stuff... \newline
    \tab Some tab stuff also...

    % equations
    \begin{equation}
        % I have used \text as to avoid actual equations
        \PP[r] = \text{obably something cool...}
    \end{equation}

    % fractions
    \begin{equation}
        \frac{1}{2}
    \end{equation}

    Hey, let's use a $\Rightarrow$ because those are cool too.
}
\item{
    
    % inline equations
    We can also do a little bit of $S_u B_s\text{cripting}$.
}
\end{enumerate}

\section{Another Title}
\begin{enumerate}[label=(\alph*)]
\item{
    For sure getting full credit on this one. 
}

\item{
    % a little something and superscripting
    \begin{equation}
        \sum_{i = 0}^{n} = \text{thing}
    \end{equation}
}
\item{
    Would you just look at this inequality $|X - 11| \geq 33$, By the Chevy Chase inequality
    this is true.
}
\end{enumerate}

\section{This One Is About Nesting Some Items And Stuff}
\begin{enumerate}[label=(\alph*)]
\item{
    \begin{enumerate}[label=(\roman*)]
    \item{
        \begin{equation*}
            2 + 2 = 4
        \end{equation*}

        \begin{equation*}
            4 + 4 = 8
        \end{equation*}
    }
    \item{
        \begin{equation*}
            4 - 3 = 1
        \end{equation*}
        But also,
        \begin{equation*}
            3 - 4 = -1
        \end{equation*}
    }
    \end{enumerate}
}
\end{enumerate}

\section{Last Title}
\begin{enumerate}[label=(\alph*)]
\item{
    Just a regular item.
}
\item{
    Given cows that say $\mu  < \epsilon$ we have that,

    \begin{equation}
        \epsilon > \text{the knights that say ni}
    \end{equation}
}
\end{enumerate}

% throw a little * on that section to get rid of
% the numbers, if you're into that kind of thing.
\section*{Wait, One More}
\begin{enumerate}[label=(\alph*)]
\item{
    I hope that this helps, \newline
    % oh yeah, you might want to include some graphics/images/whatever
    \includegraphics[width=10cm, height=10cm]{graphic.png}
    Cheers.
}
\end{enumerate}
\end{document}
